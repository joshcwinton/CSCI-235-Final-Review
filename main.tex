\documentclass{article}
\usepackage[utf8]{inputenc}

\title{CSCI 235 Final Review (Part 1: Slides)}
\author{Josh Winton}
\date{December 2018}

\begin{document}

\maketitle

\section{Introduction}
\subsection{OOP Flash Review}
\begin{itemize}
    \item Classes group together variables (data) and functions (operations on the data)
    \item Classes have:
    \begin{itemize}
        \item Member variables
        \item Member functions (or methods)
        \item Constructors (special function called when instance of class declared)
        \item Destructors (special function called when instance of class is deleted)
    \end{itemize}
    \item Instances of a class are called objects
    \item Think of classes as types and objects as variables
    \item Class members can be declared \verb|public| or \verb|private|
    \item \verb|public| members are the interface for client code interaction
    \item \verb|private| members are hidden from other code and can only be used within the class
    \item Put interface into a \verb|.h| or \verb|.hpp| file and implementation into a \verb|.cpp| file.
\end{itemize}

\subsection{Control Interaction}
\begin{itemize}
    \item Pass by value:\\ \verb|bool my_method(int some_int);|
    \item Pass by reference when you need to modify the object:\\ \verb|bool my_method(ObjectType& some_object);|
    \item Pass by constant reference if function doesn't modify object but you still want to call by reference:\\ \verb|bool my_method(const ObjectType& some_object);|
\end{itemize}

\section{Abstraction and OOP}
\subsection{Abstraction}
\begin{itemize}
    \item Abstraction and Information Hiding are a means for entirely containing complexity
    \item Implementation detail is abstracted away into a small set of commands executed by means of an interface
    \item Information hiding prevents parts of a program from inadvertently (or deliberately) modifying implementations in unexpected ways
\end{itemize}
\subsection{Object Oriented Programming}
\begin{itemize}
    \item Object Oriented Analysis and Design
    \begin{itemize}
        \item Object-oriented analysis specifies what to do, not how to do it
        \item Object-oriented design specifies software objects and their collaboration
    \end{itemize}
    \item Object Oriented Solution
    \begin{itemize}
        \item Create a good set of self contained units of code
        \item Use classes of objects to combine attributes and behaviors (data members and member functions 
    \end{itemize}
    \item Principles of Object Oriented Programming 
    \begin{itemize}
        \item Encapsulation: Objects combine data and operations
        \item Information Hiding: Objects hide inner details
        \item Inheritance: Objects inherit properties from other objects
        \item Polymorphism: Objects determine appropriate operations at execution
    \end{itemize}
    \item Operation contract specifies input and output and pre- and postconditions
\end{itemize}

\pagebreak

\section{Abstract Data Types and Templates}
\subsection{Constructors}
\begin{itemize}
    \item Declaration Example:
    \begin{verbatim}
class SomeClass
{
    public:
        SomeClass();               //default constructor
        SomeClass(parameter_list); //parameterized constructor
    
    private:
        //private members go here
};
    \end{verbatim}
    \item Implementation Example:
    \begin{verbatim}  
//default constructor
SomeClass::SomeClass();
member_var1_(initial value),
member_var2_(initial value)
{
}

//parameterized constructor
SomeClass::SomeClass(type parameter_1, type parameter_2):

member_var1(parameter_1), member_var2(parameter_2)

{

}
\end{verbatim}
    \item The default constructor is automatically supplied by the compiler if not defined
    \item If only the parameterized constructor is provided, the compiler won't supply a default constructor and the class must be initialized with parameters
\end{itemize}

\subsection{Destructors}
\begin{itemize}
    \item Default destructors are automatically supplied by the compiler if not provided by the programmer
    \item When a class uses dynamic memory allocation a destructor must be provided by the programmer
    \item The destructor declaration is \verb|~SomeClass();|
\end{itemize}

\subsection{Templates}
\begin{itemize}
    \item Useful when implementing an abstract data type (like Bag) without locking the type
    \item Vector is a template because you can do \verb|vector<type>|
    \item Templates are declared as:
    \begin{verbatim}
template<class ItemType>
class Bag
{
    //class declaration here
}
    \end{verbatim}
    \item Templates are defined as:
    \begin{verbatim}
template<class ItemType>
bool Bag<ItemType>::add(const ItemType& newEntry){
    //implementation here
}
    \end{verbatim}
    \item Instantiation (think of \verb|vector<int>|):
    \begin{verbatim}
Bag<string> stringBag;
    \end{verbatim}
    \item Always include \verb|.cpp| in the \verb|.h| file
\end{itemize}

\subsection{Inheritance}
\begin{itemize}
    \item \verb|const| after function name means that it won't modify the object
    \item \verb|virtual| and \verb|=0| allow declaration without definition, forcing the derived class to define the function
    \item Inheritance Example:
    \begin{verbatim}
class BatchPrinter: public Printer // inherit from printer
{

public: 

    //Constructor, destructor

    void addDocument(const string& document); 
    void printAllDocuments();
private:

    vector<string> documents;

}; //end BatchPrinter
    \end{verbatim}
    \item Overloading is define a new function with the same name but a different parameter list
    \item Overriding is rewriting a function with the same signature for a derived class, i.e.:
    \begin{verbatim}
//original function
int BaseClass::someMethod(){} 

//derived class, overriding function
int DerivedClass::someMethod(){} 
    \end{verbatim}
    \item Derived classes inherit all \verb|public| and \verb|protected| members of base class
    \item Derived classes have no access to private members
    \item 
\end{itemize}
\end{document}
